
\section{Installation}
\subsection{System requirements}
The binaries and the source code have been tested on 64-bit Ubuntu 14.14 and on 64-bit OSX 10.10.7 and are expected to work on any newer system equipped with Python 2.7 or greater. No particular memory requirements or processor capabilities are needed. 

\subsection{Conda Package}
Conda (\url{http://conda.pydata.org/docs/}) is a free, multi-platform and multi-language package and environment manager. Using Conda, it is possible to easily obtain molSimplify and all the environmental dependencies in one step. This is the recommended method for end users to acquire molSimplify. In order to acquire molSimplify using Conda, follow these directions:

\begin{enumerate}
\item Ensure you have an up-to-date version of Conda installed. Binary installers are available for all platforms from \url{https://www.continuum.io/downloads} - we recommend using Anaconda (as opposed to Miniconda).
\item Ensure that you have an up-to-date version of the Anaconda Cloud Client by running
\begin{lstlisting}[language=bash]
  $ conda list -n anaconda-client
\end{lstlisting} 
If the package is not found,  you need to install it (we recommend running this command as SU on Linux systems):
\begin{lstlisting}[language=bash]
  $ sudo conda install anaconda-client
\end{lstlisting} 

\item Now you are ready to install molSimplify by running:
\begin{lstlisting}[language=bash]
  $ conda install -c hjkgroup molSimplify
\end{lstlisting} 
Here, '-c' means that the files will come from the Anaconda Cloud, and hjkgroup is our repository there. Don't forget the capital S! This single command will bring openbabel with python bindings with it, if they are missing. 
\item If you want to use the GUI (graphical user interface), you'll need PyQt5, which you can get (if you don't have it) with 
\begin{lstlisting}[language=bash]
  $ conda install -c dsdale24 pyqt5
\end{lstlisting} 
\item Now you are ready to go! Launch molSimplify from the terminal using:
\begin{lstlisting}[language=bash]
  $ molsimplify 
\end{lstlisting} 
You can add any arguments here in the normal manner, for example:
\begin{lstlisting}[language=bash]
  $ molsimplify -core Fe -lig water -ligocc
\end{lstlisting} 
would generate a 6-coordinated iron-hexaaqua complex.
\end{enumerate}


While Conda remains our recommended method, other installation options exist:

\subsection{From source}
In order to install molSimplify from source it is necessary to install openbabel with its Python bindings (pybel) and numpy. Furthermore, additional software is required in order to use the Graphical User Interface (GUI).

\subsubsection{Linux}

For Linux Ubuntu, the only requirements are a working version of Openbabel with the Python bindings (pybel) and Numpy installed. To install numpy, you can follow the online instructions or use pip install numpy. We suggest installing pybel from source using the following procedure:
\begin{enumerate}
\item   wget \url{https://sourceforge.net/projects/openbabel/files/openbabel/2.3.2/openbabel-2.3.2.tar.gz}
\item  tar -xvf openbabel-2.3.2.tar.gz 
\item mkdir build ; cd build
\item cmake .. -DPYTHON\_BINDINGS=ON
\item sudo make
\item sudo make install
\end{enumerate}.

In order to enable the GUI, the library PyQt5 with its dependencies needs to be installed first. In order to do that we suggest the following procedure:
\begin{enumerate}
\item Download and install Qt5 from source (\url{developer.qt.nokia.com/}). 
\item Install SIP(\url{http://pyqt.sourceforge.net/Docs/sip4/installation.html}) from source. If Python.h is missing use \texttt{sudo apt-get install python-dev} . 
\item Install PyQt5 from source (\url{http://downloads.sourceforge.net/project/pyqt/PyQt5/}).
\end{enumerate}

Once you have installed PyQt5, open an interactive Python terminal window and check whether \begin{lstlisting}[language=python] import PyQt5 \end{lstlisting} works. \\

Once you have installed Pybel (and optionally PyQt5), you proceed with the installation. Obtain the source files from \url{https://github.com/hjkgrp/molSimplify/tree/JP}. Uncompress the zipped directory, and then navigate to top level directory in the unzipped structure. The program can then be installed with
\begin{lstlisting}[language=bash]
  $ sudo python setup.py install
\end{lstlisting} 
The code can then be used from the command line as described in the Conda section.

\subsubsection{Mac OSX}

In order to run molSimplify from source, we need to install pybel first. We suggest using \texttt{brew} as explained in section \ref{osxbin} or compiling from source. Then install imagemagick (see section \ref{osxbin}). In order to use the GUI, PyQt5 and its dependencies need to be installed as well. The easiest way is to use the following procedure:

\begin{enumerate}
\item Install Qt5 with: \texttt{sudo brew install qt5}
\item Install SIP with: \texttt{sudo brew install sip}
\item Install PyQt5 from source (\url{http://downloads.sourceforge.net/project/pyqt/PyQt5/}) with option -\hspace{0.02cm}-sip-incdir=/usr/local/include
\end{enumerate}

Once the correct dependenices are installed, you can follow the same setup instructions as in the Linux case.

\subsection{Binaries}

Binaries of the program have been compiled for both Ubuntu Linux and OSX. The installation of the binaries depends on the platform and analytical instructions are given below.

\subsubsection{Linux}\label{linbin}

For Linux Ubuntu, the only requirement is a working version of Openbabel with the Python bindings (pybel) installed. We suggest installing pybel from source using the following procedure:
\begin{enumerate}
\item   wget \url{https://sourceforge.net/projects/openbabel/files/openbabel/2.3.2/openbabel-2.3.2.tar.gz}
\item  tar -xvf openbabel-2.3.2.tar.gz 
\item mkdir build ; cd build
\item cmake .. -DPYTHON\_BINDINGS=ON
\item sudo make
\item sudo make install
\end{enumerate}

Alternatively pybel can be installed using the standard ubuntu package manager with \texttt{apt-get install python-openbabel}.

In both cases, once the installation is complete, open an interactive Python terminal and make sure that \texttt{import openbabel} and \texttt{import pybel} work before running molSimplify. 

With pybel installed, the user can download the binary and the supporting files and run the program. The program will guide the user in order to set up a configuration file that is stored under \texttt{/home/user/.molSimplify} in the user's home directory and contains a line that specifies the installation directory (top directory) of the program where all the supporting files are located. The line has the format \texttt{INSTALLDIR=path}. The binary can be moved anywhere, however if the top directory of molSimplify is moved the \texttt{.molSimplify} file will have to be manually updated. The configuration file contains also the paths for files used in database search and post processing analysis (see section \ref{dbpp}).

\subsubsection{Mac OSX}\label{osxbin}

For OSX, a package installer has been created that automatically installs the required software for running the molSimplify app. The user can open the \texttt{molSimplify.pkg} installer, follow the instructions and get molSimplify installed.

If the package installer fails, the user can manually install the required packages and then download a copy of the molSimplify app. Two packages are required in order to run the molSimplify app, pybel and imagemagick. The easiest way to install both is to use the \texttt{brew} package manager and issue the following commands on a terminal window:
\begin{enumerate}
\item sudo brew install open-babel -\hspace{0.02cm}-with-python
\item sudo brew install imagemagick
\end{enumerate}

With pybel installed, the user can download the app, place it under /Applications/molSimplify.app and run the program.

If \texttt{brew} is not installed, install it by running \texttt{/usr/bin/ruby -e "\$(curl -fsSL https://raw.githubusercontent.com/Homebrew/install/master/install)"} in a terminal window.



\subsection{Installed files}

With both the binaries or the source code a set of additional files are stored that are required by molSimplify. If you use molSimplify with the molSimplify.app (OSX) these files are stored under /Applications/molSimplify.app/Contents/Resources/, otherwise they are in the installation directory of molSimplify as indicated in the \texttt{.molSimplify} file in the home directory. These additional files include the following:
\begin{itemize}
\item icons: Folder containing the icons used in the GUI.
\item Data: Folder containing the available coordinations/geometries and Metal-Ligand bond length database.
\item Cores: Folder containing the local saved cores.
\item Ligands: Folder containing the local saved ligands.
\item Bind: Folder containing the local saved extra molecules.
\end{itemize}

You can manually update the contents of these folders, however it is recommended to use the modules included in the GUI in order to update them. \\

In the Conda version, package resource management is handled by Conda and Python, and you don't need to worry about them.

\subsection{Databases \& Post-processing}\label{dbpp}

\subsubsection{Chemical database setup}


The program is able to interact with external chemical databases such as ChEMBL. These databases are essentially sdf files that contain information about molecules. In order to link these databases to molSimplify the user needs first of all to download the database sdf file. 

Once the file is downloaded, we recommend using Openbabel's fast search index generation by issuing  \texttt{babel DBfile.sdf -ofs}. Openbabel will then create a .fs file that can be used for fast screening and will substantially accelerate the database search. Note that OpenBabel can handle sdf files smaller than 4GB, so you may have to break larger files before fast indexing them. 

All the sdf and fs files should be placed inside a common directory. Then by clicking on the search DB window on the GUI, a prompt will ask you to specify that directory. Once you specify the directory containing your sdf and fs files, molSimplify will automatically detect them and the database search will be ready to use. Alternatively the user can specify the path by modifying the \texttt{.molSimplify} file in the home directory and adding a line with \texttt{CHEMDBDIR=path}. 

A short list of available chemical databases is:
\begin{itemize}
\item ChEMBL \url{https://www.ebi.ac.uk/chembl/}
\item eMolecules \url{https://www.emolecules.com/}
\item ChEBI \url{http://www.ebi.ac.uk/chebi/}
\item Zinc \url{http://zinc.docking.org/}
\item PubChem \url{http://pubchem.ncbi.nlm.nih.gov/}
\end{itemize}

\subsubsection{Multiwfn setup}

Many of the features included in the post processing module of molSimplify depend on the external program Multiwfn. The user needs to manually install Multiwfn from \url{https://multiwfn.codeplex.com/} and then specify the installation directory by clicking the post processing option in the GUI and then following the prompt instructions.  Alternatively the user can specify the path to the Multiwfn executable by modifying the \texttt{.molSimplify} file in the home directory and adding a line with \texttt{MULTIWFN=path}.
